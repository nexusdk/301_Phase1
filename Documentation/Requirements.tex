%
\documentclass[12pt,a4paper]{article}
\begin{document}

\title{Requirements Document}
\author{Neels van Rooyen u29052735\\
Luan van der Weshuizen u10134043\\
Ndivhuwo Nthambeleni u10001183\\
Cebolenkosi Makeleni u10534505\\
Dieter Doman u11002566\\
Christopher Crossman u10134842}
\maketitle
\pagebreak
\begin{center}
\subsection*{Revisions}
\begin{tabular}{|p{2cm}|p{4cm}|p{4cm}|p{2cm}|}
\hline 
Version & Primary author(s) & Description of version & Date complete \\\hline
v1.0 & • & • & • \\ 
\hline 
• & • & • & • \\ 
\hline 
• & • & • & • \\ 
\hline 
• & • & • & • \\ 
\hline 
\end{tabular} \\
\end{center}
\begin{center}
\subsection*{Review}
\paragraph{Requirements document review history}
\begin{tabular}{|p{4cm}|p{4cm}|p{2cm}|p{2cm}|}
\hline 
Approving party & Version approved & Signature & Date \\\hline
• & • & • & • \\ 
\hline 
• & • & • & • \\ 
\hline 
• & • & • & • \\ 
\hline
\end{tabular} 
\end{center}
\section{The Purpose of the Project}
\subsection{Purpose}
\paragraph{The purpose of this document is to make an agreement between the developers and the client Mr Jan Kroeze from the Department of Computer Science about the requirements of the new Marks System that needs to be developed. This document is to ensure the scope of the project is correct and clear to all the stakeholders in this project. This document the basis for which further phases will be built on.}
\subsection{Project Scope}
\paragraph{The scope of the project is to develope a software solution which reduces the paper work of marking students of a course in the Computer Science Department for lectures and markers. It also will uphold the privicy of the students so that other students can't see their mark. Reducing the paper work reduces the chances that marks can be lost.}
\subsection{Document conventions}
\begin{itemize}
\item Documentation formulation: LaTeX
\item UML: Class diagrams and activity diagrams
\end{itemize}
\pagebreak
\section{The Stakeholders}
\subsection{The Client}
\subsection{The Customer}
\subsection{Other Stakeholders}
\subsection{The Hands-On Users of the Product}
\subsection{Personas}
\subsection{Priorities Assigned to Users}
\subsection{User Participation}
\subsection{Maintenance Users and Service Technicians}
\pagebreak
\section{Mandated Constraints}
\subsection{Solution Constraints}
\paragraph{
This specifies constraints on the way that the problem must be solved. Describe the mandated technology or solution. You should also explain the reason for using the technology. The constraints are treated as a type of requirement.}
\subsection{Implementation Environment of the Current System}
\paragraph{The current system will be implemented in different environments. The first being in the form of a web service whereby lectures can access information about marks and assessments of students, and students can access their marks individually by going on to the website and entering the user name and password.
The second environment being smartphones through an android application meant to be used by tutors and teaching assistants to record marks, and also by lectures in special cases such as helping tutors and/or teaching assistants with marking.}
\subsection{Partner or Collaborative Applications}
\subsection{Off-the-Shelf Software}
\subsection{Anticipated Workplace Environment}
\subsection{Schedule Constraints}
\pagebreak
\section{Naming Conventions and Terminology}
\subsection{Definitions of All Terms, Including Acronyms, Used by Stakeholders involved in the Project}
\begin{itemize}
\item UML - Unified Modeling Language
\item LDAP - Lightweight Directory Access Protocol
\item CSV - Comma-separated values
\item Andriod - Is an operating system based on the Linux kernel
\item Practical - A booked session where practicals are displayed and question can be asked about the weeks practical
\item Authentication - Grant access if details are legitimate
\item Lecture - A person who gives lectures and manages the course
\item Marker - Person that evaluates the practicals
\end{itemize}
\pagebreak
\section{Relevant Facts and Assumptions}
\subsection{Relevant Facts}
\subsection{Business Rules}
\subsection{Assumptions}
\paragraph{For this product we assume lecturers, tutors and teaching assistant have access to android enabled devices or alternatively internet and web enabled devices to use the product.
We assume students will have access to the resources required to use the product.
}
\pagebreak
\section{The Scope of the Work}
\subsection{The Current Situation}
\subsection{The Context of the Work}
\subsection{Work Partitioning}
\subsection{Specifying a Business Use Case}
\pagebreak
\section{The Scope of the Product}
\subsection{Product Boundary}
\paragraph{}
A use case diagram identifies the boundaries between the users (actors) and the product you are about to build (this is sometimes called "the system"). You arrive at the product boundary by inspecting each business use case and determining, in conjunction with the appropriate stakeholders, which part of the business use case should be automated (or satisfied by some sort of product) and what part should be done by the user. This task must take into account the abilities of the actors (section 2), the constraints (section 3), the goals of the project (section 1), and your knowledge of both the work and the technology that can make the best contribution to the work.
The use case diagram shows the actors/users outside the product boundary (the rectangle). The product use cases (PUCs) are the ellipses inside the boundary. The numbers link each PUC back to the BUC that it came from (see section 6). The lines denote usage. Note that actors can be either automated or human.
\paragraph{}
Derive the PUCs by deciding where the product boundary should be for each business use case (BUC). These decisions are based on your knowledge of the work and the requirements constraints. Note that the PUCs that you come up with must be traceable back to the BUCs. The numbers on the PUC diagram correspond to the BUC numbers on the Business Event List (see section 6).
\subsection{Product Use Case Table}
\subsection{Individual Product Use Cases}
\pagebreak
\section{Functional Requirements}
\paragraph{}
Functional requirements are the fundamental or essential subject matter of the product. They describe what the product has to do or what processing actions it is to take.
\subsection{Functional Requirements}
\subsection{Requirements Shell}
\paragraph{}
Each atomic requirement is made up of a number of attributes. The requirements shell is a guide to writing each atomic functional (section 9) and non-functional requirements (sections 10-17). Here is an example of the content of the shell shown in graphic form. The shell can and should be automated.
\pagebreak
\section{Non-functional Requirements}
\subsection{Look and Feel Requirements}
\paragraph{}
Nonfunctional requirements (sections 10-17) are the properties that the functions must have, such as performance and usability. Do not be deterred by the unfortunate type name (we use it because it is the most common way of referring to these types of requirements)—these requirements are as important as the functional requirements for the product’s success.
\subsubsection{Appearance Requirements}
\subsubsection{Style Requirements}
\subsection{Usability and Humanity Requirements}
\subsubsection{Ease of Use Requirements}
\subsubsection{Personalization Requirements}
\subsubsection{Learning Requirements}
\paragraph{}
How easy it is to learn how to use the product.
\subsubsection{Understandability and Politeness Requirements}
\subsubsection{Accessibility Requirements}
\paragraph{}
The requirements for how easy it should be for people with common disabilities to access the product. These disabilities might be related to physical disability or visual, hearing, cognitive, or other abilities.
\pagebreak
\section{Performance Requirements}
\subsection{Speed and Latency Requirements}
\subsection{Safety-Critical Requirements}
\subsection{Precision or Accuracy Requirements}
\subsection{Reliability and Availability Requirements}
\subsection{Robustness or Fault-Tolerance Requirements}
\subsection{Capacity Requirements}
\subsection{Scalability or Extensibility Requirements}
\subsection{Longevity Requirements}
\pagebreak
\section{Operational and Environmental Requirements}
\subsection{Expected Physical Environment}
\subsection{Requirements for Interfacing with Adjacent Systems}
\subsection{Release Requirements}
\pagebreak
\section{Maintainability and Support Requirements}
\subsection{Maintenance Requirements}
\subsection{Supportability Requirements}
\subsection{Adaptability Requirements}
\pagebreak
\section{Security Requirements}
\subsection{Access Requirements}
\subsection{Integrity Requirements}
\subsection{Privacy Requirements}
\subsection{Audit Requirements}
\subsection{Immunity Requirements}
\pagebreak
\section{Legal Requirements}
\subsection{Compliance Requirements}
\subsection{Standards Requirements}
\pagebreak
\section{Off-the-Shelf Solutions}
\pagebreak
\section{New Problems}
\subsection{Effects on the Current Environment}
\subsection{Effects on Installed Systems}
\subsection{Potential User Problems}
\subsection{Limitations in the Anticipated Implementation Environment That May Inhibit the New Product}
\subsection{Follow-Up Problems}
\pagebreak
\section{Migration to the New Product}
\subsection{Data That Has to Be Modified or Translated for the New System}
\pagebreak
\section{Risks}
\end{document}
